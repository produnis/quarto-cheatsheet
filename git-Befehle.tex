% Options for packages loaded elsewhere
\PassOptionsToPackage{unicode}{hyperref}
\PassOptionsToPackage{hyphens}{url}
\PassOptionsToPackage{dvipsnames,svgnames,x11names}{xcolor}
%
\documentclass[
]{article}

\usepackage{amsmath,amssymb}
\usepackage{iftex}
\ifPDFTeX
  \usepackage[T1]{fontenc}
  \usepackage[utf8]{inputenc}
  \usepackage{textcomp} % provide euro and other symbols
\else % if luatex or xetex
  \usepackage{unicode-math}
  \defaultfontfeatures{Scale=MatchLowercase}
  \defaultfontfeatures[\rmfamily]{Ligatures=TeX,Scale=1}
\fi
\usepackage{lmodern}
\ifPDFTeX\else  
    % xetex/luatex font selection
\fi
% Use upquote if available, for straight quotes in verbatim environments
\IfFileExists{upquote.sty}{\usepackage{upquote}}{}
\IfFileExists{microtype.sty}{% use microtype if available
  \usepackage[]{microtype}
  \UseMicrotypeSet[protrusion]{basicmath} % disable protrusion for tt fonts
}{}
\makeatletter
\@ifundefined{KOMAClassName}{% if non-KOMA class
  \IfFileExists{parskip.sty}{%
    \usepackage{parskip}
  }{% else
    \setlength{\parindent}{0pt}
    \setlength{\parskip}{6pt plus 2pt minus 1pt}}
}{% if KOMA class
  \KOMAoptions{parskip=half}}
\makeatother
\usepackage{xcolor}
\usepackage[a4paper,landscape,bottom=78mm]{geometry}
\setlength{\emergencystretch}{3em} % prevent overfull lines
\setcounter{secnumdepth}{-\maxdimen} % remove section numbering
% Make \paragraph and \subparagraph free-standing
\ifx\paragraph\undefined\else
  \let\oldparagraph\paragraph
  \renewcommand{\paragraph}[1]{\oldparagraph{#1}\mbox{}}
\fi
\ifx\subparagraph\undefined\else
  \let\oldsubparagraph\subparagraph
  \renewcommand{\subparagraph}[1]{\oldsubparagraph{#1}\mbox{}}
\fi


\providecommand{\tightlist}{%
  \setlength{\itemsep}{0pt}\setlength{\parskip}{0pt}}\usepackage{longtable,booktabs,array}
\usepackage{calc} % for calculating minipage widths
% Correct order of tables after \paragraph or \subparagraph
\usepackage{etoolbox}
\makeatletter
\patchcmd\longtable{\par}{\if@noskipsec\mbox{}\fi\par}{}{}
\makeatother
% Allow footnotes in longtable head/foot
\IfFileExists{footnotehyper.sty}{\usepackage{footnotehyper}}{\usepackage{footnote}}
\makesavenoteenv{longtable}
\usepackage{graphicx}
\makeatletter
\def\maxwidth{\ifdim\Gin@nat@width>\linewidth\linewidth\else\Gin@nat@width\fi}
\def\maxheight{\ifdim\Gin@nat@height>\textheight\textheight\else\Gin@nat@height\fi}
\makeatother
% Scale images if necessary, so that they will not overflow the page
% margins by default, and it is still possible to overwrite the defaults
% using explicit options in \includegraphics[width, height, ...]{}
\setkeys{Gin}{width=\maxwidth,height=\maxheight,keepaspectratio}
% Set default figure placement to htbp
\makeatletter
\def\fps@figure{htbp}
\makeatother

\usepackage{tikz}
\usepackage{url}
\usepackage{multicol}
\usepackage{amsmath}
\usepackage{esint}
\usepackage{amsfonts}
\usepackage{tikz}
\usetikzlibrary{decorations.pathmorphing}
\usepackage{amsmath,amssymb}

\usepackage{colortbl}
\usepackage{xcolor}
\usepackage{mathtools}
\usepackage{amsmath,amssymb}
\usepackage{enumitem}
\makeatletter

\newcommand*\bigcdot{\mathpalette\bigcdot@{.5}}
\newcommand*\bigcdot@[2]{\mathbin{\vcenter{\hbox{\scalebox{#2}{$\m@th#1\bullet$}}}}}
\makeatother

\title{$title$}

\advance\topmargin-.8in
\advance\textheight3in
\advance\textwidth3in
\advance\oddsidemargin-1.5in
\advance\evensidemargin-1.5in
\parindent0pt
\parskip2pt
\newcommand{\hr}{\centerline{\rule{3.5in}{1pt}}}
\makeatletter
\makeatother
\makeatletter
\makeatother
\makeatletter
\@ifpackageloaded{caption}{}{\usepackage{caption}}
\AtBeginDocument{%
\ifdefined\contentsname
  \renewcommand*\contentsname{Inhaltsverzeichnis}
\else
  \newcommand\contentsname{Inhaltsverzeichnis}
\fi
\ifdefined\listfigurename
  \renewcommand*\listfigurename{Abbildungsverzeichnis}
\else
  \newcommand\listfigurename{Abbildungsverzeichnis}
\fi
\ifdefined\listtablename
  \renewcommand*\listtablename{Tabellenverzeichnis}
\else
  \newcommand\listtablename{Tabellenverzeichnis}
\fi
\ifdefined\figurename
  \renewcommand*\figurename{Abbildung}
\else
  \newcommand\figurename{Abbildung}
\fi
\ifdefined\tablename
  \renewcommand*\tablename{Tabelle}
\else
  \newcommand\tablename{Tabelle}
\fi
}
\@ifpackageloaded{float}{}{\usepackage{float}}
\floatstyle{ruled}
\@ifundefined{c@chapter}{\newfloat{codelisting}{h}{lop}}{\newfloat{codelisting}{h}{lop}[chapter]}
\floatname{codelisting}{Listing}
\newcommand*\listoflistings{\listof{codelisting}{Listingverzeichnis}}
\makeatother
\makeatletter
\@ifpackageloaded{caption}{}{\usepackage{caption}}
\@ifpackageloaded{subcaption}{}{\usepackage{subcaption}}
\makeatother
\makeatletter
\@ifpackageloaded{tcolorbox}{}{\usepackage[skins,breakable]{tcolorbox}}
\makeatother
\makeatletter
\@ifundefined{shadecolor}{\definecolor{shadecolor}{rgb}{.97, .97, .97}}
\makeatother
\makeatletter
\makeatother
\makeatletter
\makeatother
\ifLuaTeX
\usepackage[bidi=basic]{babel}
\else
\usepackage[bidi=default]{babel}
\fi
\babelprovide[main,import]{ngerman}
% get rid of language-specific shorthands (see #6817):
\let\LanguageShortHands\languageshorthands
\def\languageshorthands#1{}
\ifLuaTeX
  \usepackage{selnolig}  % disable illegal ligatures
\fi
\IfFileExists{bookmark.sty}{\usepackage{bookmark}}{\usepackage{hyperref}}
\IfFileExists{xurl.sty}{\usepackage{xurl}}{} % add URL line breaks if available
\urlstyle{same} % disable monospaced font for URLs
\hypersetup{
  pdftitle={Git Befehle},
  pdflang={de},
  colorlinks=true,
  linkcolor={black},
  filecolor={Maroon},
  citecolor={Blue},
  urlcolor={Blue},
  pdfcreator={LaTeX via pandoc}}

\title{Git Befehle}
\author{}
\date{}

\begin{document}
\begin{center}{\huge{\textbf{Git Befehle}}}\\
\end{center}
\begin{multicols*}{3}

\definecolor{HSNR1}{RGB}{024, 081, 145}
\tikzstyle{mybox} = [draw=HSNR1, fill=white, very thick,
    rectangle, rounded corners, inner sep=10pt, inner ysep=11pt, text width=0.3\textwidth]
\tikzstyle{fancytitle} =[fill=HSNR1, text=white, font=\bfseries]\ifdefined\Shaded\renewenvironment{Shaded}{\begin{tcolorbox}[frame hidden, borderline west={3pt}{0pt}{shadecolor}, sharp corners, boxrule=0pt, breakable, enhanced, interior hidden]}{\end{tcolorbox}}\fi

\hypertarget{allgemein}{%
\section{Allgemein}\label{allgemein}}

\begin{tikzpicture}
    \node [mybox, align=left] (box){%
     Ins Verzeichnis wechseln\\\texttt{git init}
    };
    %------------ Header ---------------------
    \node[fancytitle, right=10pt] at (box.north west) {Neues Git Repository};
 \end{tikzpicture}
\smallskip

\hypertarget{krefeld}{%
\subsection{Krefeld}\label{krefeld}}

\begin{tikzpicture}
    \node [mybox, align=left] (box){%
     Neuen Branch anlegen:\\\texttt{git checkout -b SoSe2024}\\ \texttt{git push –set-upstream origin SoSe2024}
    };
    %------------ Header ---------------------
    \node[fancytitle, right=10pt] at (box.north west) {Ein neues Semester};
 \end{tikzpicture}
\smallskip

\begin{tikzpicture}
    \node [mybox, align=left] (box){%
     \texttt{git checkout main}\\ \texttt{git merge SoSe2024}
    };
    %------------ Header ---------------------
    \node[fancytitle, right=10pt] at (box.north west) {Semester in \texttt{main} mergen};
 \end{tikzpicture}
\smallskip

\begin{tikzpicture}
    \node [mybox, align=left] (box){%
     Tag setzen:\\\texttt{git tag -a SoSe24 -m “Meine Lehre im Sommersemester 2024”}
    };
    %------------ Header ---------------------
    \node[fancytitle, right=10pt] at (box.north west) {Abgeschlossene Semester taggen};
 \end{tikzpicture}
\smallskip

\begin{tikzpicture}
    \node [mybox, align=left] (box){%
     alle Tags pushen:\\\smallskip\texttt{git push origin –tags} \\nur bestimmten Tag pushen:\\\texttt{git push origin SoSe24}
    };
    %------------ Header ---------------------
    \node[fancytitle, right=10pt] at (box.north west) {Tags auf den Server übertragen};
 \end{tikzpicture}
\smallskip

\hypertarget{allgemein-1}{%
\section{Allgemein}\label{allgemein-1}}

\begin{tikzpicture}
    \node [mybox, align=left] (box){%
     Ins Verzeichnis wechseln\\\texttt{git init}
    };
    %------------ Header ---------------------
    \node[fancytitle, right=10pt] at (box.north west) {Neues Git Repository};
 \end{tikzpicture}
\smallskip

\hypertarget{krefeld-1}{%
\subsection{Krefeld}\label{krefeld-1}}

\begin{tikzpicture}
    \node [mybox, align=left] (box){%
     Neuen Branch anlegen:\\\texttt{git checkout -b SoSe2024}\\ \texttt{git push –set-upstream origin SoSe2024}
    };
    %------------ Header ---------------------
    \node[fancytitle, right=10pt] at (box.north west) {Ein neues Semester};
 \end{tikzpicture}
\smallskip

\begin{tikzpicture}
    \node [mybox, align=left] (box){%
     \texttt{git checkout main}\\ \texttt{git merge SoSe2024}
    };
    %------------ Header ---------------------
    \node[fancytitle, right=10pt] at (box.north west) {Semester nach Abschluss in \texttt{main} mergen};
 \end{tikzpicture}
\smallskip

\begin{tikzpicture}
    \node [mybox, align=left] (box){%
     Tag setzen:\\\texttt{git tag -a SoSe24 -m “Meine Lehre im Sommersemester 2024”}
    };
    %------------ Header ---------------------
    \node[fancytitle, right=10pt] at (box.north west) {Abgeschlossene Semester taggen};
 \end{tikzpicture}
\smallskip

\begin{tikzpicture}
    \node [mybox, align=left] (box){%
     alle Tags pushen:\\\smallskip\texttt{git push origin –tags} \\nur bestimmten Tag pushen:\\\texttt{git push origin SoSe24}
    };
    %------------ Header ---------------------
    \node[fancytitle, right=10pt] at (box.north west) {Tags auf den Server übertragen};
 \end{tikzpicture}
\smallskip

\hypertarget{allgemein-2}{%
\section{Allgemein}\label{allgemein-2}}

\begin{tikzpicture}
    \node [mybox, align=left] (box){%
     Ins Verzeichnis wechseln\\\texttt{git init}
    };
    %------------ Header ---------------------
    \node[fancytitle, right=10pt] at (box.north west) {Neues Git Repository};
 \end{tikzpicture}
\smallskip

\hypertarget{krefeld-2}{%
\subsection{Krefeld}\label{krefeld-2}}

\begin{tikzpicture}
    \node [mybox, align=left] (box){%
     Neuen Branch anlegen:\\\texttt{git checkout -b SoSe2024}\\ \texttt{git push –set-upstream origin SoSe2024}
    };
    %------------ Header ---------------------
    \node[fancytitle, right=10pt] at (box.north west) {Ein neues Semester};
 \end{tikzpicture}
\smallskip

\begin{tikzpicture}
    \node [mybox, align=left] (box){%
     \texttt{git checkout main}\\ \texttt{git merge SoSe2024}
    };
    %------------ Header ---------------------
    \node[fancytitle, right=10pt] at (box.north west) {Semester nach Abschluss in \texttt{main} mergen};
 \end{tikzpicture}
\smallskip

\begin{tikzpicture}
    \node [mybox, align=left] (box){%
     Tag setzen:\\\texttt{git tag -a SoSe24 -m “Meine Lehre im Sommersemester 2024”}
    };
    %------------ Header ---------------------
    \node[fancytitle, right=10pt] at (box.north west) {Abgeschlossene Semester taggen};
 \end{tikzpicture}
\smallskip

\begin{tikzpicture}
    \node [mybox, align=left] (box){%
     alle Tags pushen:\\\smallskip\texttt{git push origin –tags} \\nur bestimmten Tag pushen:\\\texttt{git push origin SoSe24}
    };
    %------------ Header ---------------------
    \node[fancytitle, right=10pt] at (box.north west) {Tags auf den Server übertragen};
 \end{tikzpicture}
\smallskip

\hypertarget{allgemein-3}{%
\section{Allgemein}\label{allgemein-3}}

\begin{tikzpicture}
    \node [mybox, align=left] (box){%
     Ins Verzeichnis wechseln\\\texttt{git init}
    };
    %------------ Header ---------------------
    \node[fancytitle, right=10pt] at (box.north west) {Neues Git Repository};
 \end{tikzpicture}
\smallskip

\hypertarget{krefeld-3}{%
\subsection{Krefeld}\label{krefeld-3}}

\begin{tikzpicture}
    \node [mybox, align=left] (box){%
     Neuen Branch anlegen:\\\texttt{git checkout -b SoSe2024}\\ \texttt{git push –set-upstream origin SoSe2024}
    };
    %------------ Header ---------------------
    \node[fancytitle, right=10pt] at (box.north west) {Ein neues Semester};
 \end{tikzpicture}
\smallskip

\begin{tikzpicture}
    \node [mybox, align=left] (box){%
     \texttt{git checkout main}\\ \texttt{git merge SoSe2024}
    };
    %------------ Header ---------------------
    \node[fancytitle, right=10pt] at (box.north west) {Semester nach Abschluss in \texttt{main} mergen};
 \end{tikzpicture}
\smallskip

\begin{tikzpicture}
    \node [mybox, align=left] (box){%
     Tag setzen:\\\texttt{git tag -a SoSe24 -m “Meine Lehre im Sommersemester 2024”}
    };
    %------------ Header ---------------------
    \node[fancytitle, right=10pt] at (box.north west) {Abgeschlossene Semester taggen};
 \end{tikzpicture}
\smallskip

\begin{tikzpicture}
    \node [mybox, align=left] (box){%
     alle Tags pushen:\\\smallskip\texttt{git push origin –tags} \\nur bestimmten Tag pushen:\\\texttt{git push origin SoSe24}
    };
    %------------ Header ---------------------
    \node[fancytitle, right=10pt] at (box.north west) {Tags auf den Server übertragen};
 \end{tikzpicture}
\smallskip

\hypertarget{allgemein-4}{%
\section{Allgemein}\label{allgemein-4}}

\begin{tikzpicture}
    \node [mybox, align=left] (box){%
     Ins Verzeichnis wechseln\\\texttt{git init}
    };
    %------------ Header ---------------------
    \node[fancytitle, right=10pt] at (box.north west) {Neues Git Repository};
 \end{tikzpicture}
\smallskip

\hypertarget{krefeld-4}{%
\subsection{Krefeld}\label{krefeld-4}}

\begin{tikzpicture}
    \node [mybox, align=left] (box){%
     Neuen Branch anlegen:\\\texttt{git checkout -b SoSe2024}\\ \texttt{git push –set-upstream origin SoSe2024}
    };
    %------------ Header ---------------------
    \node[fancytitle, right=10pt] at (box.north west) {Ein neues Semester};
 \end{tikzpicture}
\smallskip

\begin{tikzpicture}
    \node [mybox, align=left] (box){%
     \texttt{git checkout main}\\ \texttt{git merge SoSe2024}
    };
    %------------ Header ---------------------
    \node[fancytitle, right=10pt] at (box.north west) {Semester nach Abschluss in \texttt{main} mergen};
 \end{tikzpicture}
\smallskip

\begin{tikzpicture}
    \node [mybox, align=left] (box){%
     Tag setzen:\\\texttt{git tag -a SoSe24 -m “Meine Lehre im Sommersemester 2024”}
    };
    %------------ Header ---------------------
    \node[fancytitle, right=10pt] at (box.north west) {Abgeschlossene Semester taggen};
 \end{tikzpicture}
\smallskip

\begin{tikzpicture}
    \node [mybox, align=left] (box){%
     alle Tags pushen:\\\smallskip\texttt{git push origin –tags} \\nur bestimmten Tag pushen:\\\texttt{git push origin SoSe24}
    };
    %------------ Header ---------------------
    \node[fancytitle, right=10pt] at (box.north west) {Tags auf den Server übertragen};
 \end{tikzpicture}
\smallskip

\hypertarget{allgemein-5}{%
\section{Allgemein}\label{allgemein-5}}

\begin{tikzpicture}
    \node [mybox, align=left] (box){%
     Ins Verzeichnis wechseln\\\texttt{git init}
    };
    %------------ Header ---------------------
    \node[fancytitle, right=10pt] at (box.north west) {Neues Git Repository};
 \end{tikzpicture}
\smallskip

\hypertarget{krefeld-5}{%
\subsection{Krefeld}\label{krefeld-5}}

\begin{tikzpicture}
    \node [mybox, align=left] (box){%
     Neuen Branch anlegen:\\\texttt{git checkout -b SoSe2024}\\ \texttt{git push –set-upstream origin SoSe2024}
    };
    %------------ Header ---------------------
    \node[fancytitle, right=10pt] at (box.north west) {Ein neues Semester};
 \end{tikzpicture}
\smallskip

\begin{tikzpicture}
    \node [mybox, align=left] (box){%
     \texttt{git checkout main}\\ \texttt{git merge SoSe2024}
    };
    %------------ Header ---------------------
    \node[fancytitle, right=10pt] at (box.north west) {Semester nach Abschluss in \texttt{main} mergen};
 \end{tikzpicture}
\smallskip

\begin{tikzpicture}
    \node [mybox, align=left] (box){%
     Tag setzen:\\\texttt{git tag -a SoSe24 -m “Meine Lehre im Sommersemester 2024”}
    };
    %------------ Header ---------------------
    \node[fancytitle, right=10pt] at (box.north west) {Abgeschlossene Semester taggen};
 \end{tikzpicture}
\smallskip

\begin{tikzpicture}
    \node [mybox, align=left] (box){%
     alle Tags pushen:\\\smallskip\texttt{git push origin –tags} \\nur bestimmten Tag pushen:\\\texttt{git push origin SoSe24}
    };
    %------------ Header ---------------------
    \node[fancytitle, right=10pt] at (box.north west) {Tags auf den Server übertragen};
 \end{tikzpicture}
\smallskip

\hypertarget{allgemein-6}{%
\section{Allgemein}\label{allgemein-6}}

\begin{tikzpicture}
    \node [mybox, align=left] (box){%
     Ins Verzeichnis wechseln\\\texttt{git init}
    };
    %------------ Header ---------------------
    \node[fancytitle, right=10pt] at (box.north west) {Neues Git Repository};
 \end{tikzpicture}
\smallskip

\hypertarget{krefeld-6}{%
\subsection{Krefeld}\label{krefeld-6}}

\begin{tikzpicture}
    \node [mybox, align=left] (box){%
     Neuen Branch anlegen:\\\texttt{git checkout -b SoSe2024}\\ \texttt{git push –set-upstream origin SoSe2024}
    };
    %------------ Header ---------------------
    \node[fancytitle, right=10pt] at (box.north west) {Ein neues Semester};
 \end{tikzpicture}
\smallskip

\begin{tikzpicture}
    \node [mybox, align=left] (box){%
     \texttt{git checkout main}\\ \texttt{git merge SoSe2024}
    };
    %------------ Header ---------------------
    \node[fancytitle, right=10pt] at (box.north west) {Semester nach Abschluss in \texttt{main} mergen};
 \end{tikzpicture}
\smallskip

\begin{tikzpicture}
    \node [mybox, align=left] (box){%
     Tag setzen:\\\texttt{git tag -a SoSe24 -m “Meine Lehre im Sommersemester 2024”}
    };
    %------------ Header ---------------------
    \node[fancytitle, right=10pt] at (box.north west) {Abgeschlossene Semester taggen};
 \end{tikzpicture}
\smallskip

\begin{tikzpicture}
    \node [mybox, align=left] (box){%
     alle Tags pushen:\\\smallskip\texttt{git push origin –tags} \\nur bestimmten Tag pushen:\\\texttt{git push origin SoSe24}
    };
    %------------ Header ---------------------
    \node[fancytitle, right=10pt] at (box.north west) {Tags auf den Server übertragen};
 \end{tikzpicture}
\smallskip

\hypertarget{allgemein-7}{%
\section{Allgemein}\label{allgemein-7}}

\begin{tikzpicture}
    \node [mybox, align=left] (box){%
     Ins Verzeichnis wechseln\\\texttt{git init}
    };
    %------------ Header ---------------------
    \node[fancytitle, right=10pt] at (box.north west) {Neues Git Repository};
 \end{tikzpicture}
\smallskip

\hypertarget{krefeld-7}{%
\subsection{Krefeld}\label{krefeld-7}}

\begin{tikzpicture}
    \node [mybox, align=left] (box){%
     Neuen Branch anlegen:\\\texttt{git checkout -b SoSe2024}\\ \texttt{git push –set-upstream origin SoSe2024}
    };
    %------------ Header ---------------------
    \node[fancytitle, right=10pt] at (box.north west) {Ein neues Semester};
 \end{tikzpicture}
\smallskip

\begin{tikzpicture}
    \node [mybox, align=left] (box){%
     \texttt{git checkout main}\\ \texttt{git merge SoSe2024}
    };
    %------------ Header ---------------------
    \node[fancytitle, right=10pt] at (box.north west) {Semester nach Abschluss in \texttt{main} mergen};
 \end{tikzpicture}
\smallskip

\begin{tikzpicture}
    \node [mybox, align=left] (box){%
     Tag setzen:\\\texttt{git tag -a SoSe24 -m “Meine Lehre im Sommersemester 2024”}
    };
    %------------ Header ---------------------
    \node[fancytitle, right=10pt] at (box.north west) {Abgeschlossene Semester taggen};
 \end{tikzpicture}
\smallskip

\begin{tikzpicture}
    \node [mybox, align=left] (box){%
     alle Tags pushen:\\\smallskip\texttt{git push origin –tags} \\nur bestimmten Tag pushen:\\\texttt{git push origin SoSe24}
    };
    %------------ Header ---------------------
    \node[fancytitle, right=10pt] at (box.north west) {Tags auf den Server übertragen};
 \end{tikzpicture}
\smallskip

\hypertarget{allgemein-8}{%
\section{Allgemein}\label{allgemein-8}}

\begin{tikzpicture}
    \node [mybox, align=left] (box){%
     Ins Verzeichnis wechseln\\\texttt{git init}
    };
    %------------ Header ---------------------
    \node[fancytitle, right=10pt] at (box.north west) {Neues Git Repository};
 \end{tikzpicture}
\smallskip

\hypertarget{krefeld-8}{%
\subsection{Krefeld}\label{krefeld-8}}

\begin{tikzpicture}
    \node [mybox, align=left] (box){%
     Neuen Branch anlegen:\\\texttt{git checkout -b SoSe2024}\\ \texttt{git push –set-upstream origin SoSe2024}
    };
    %------------ Header ---------------------
    \node[fancytitle, right=10pt] at (box.north west) {Ein neues Semester};
 \end{tikzpicture}
\smallskip

\begin{tikzpicture}
    \node [mybox, align=left] (box){%
     \texttt{git checkout main}\\ \texttt{git merge SoSe2024}
    };
    %------------ Header ---------------------
    \node[fancytitle, right=10pt] at (box.north west) {Semester nach Abschluss in \texttt{main} mergen};
 \end{tikzpicture}
\smallskip

\begin{tikzpicture}
    \node [mybox, align=left] (box){%
     Tag setzen:\\\texttt{git tag -a SoSe24 -m “Meine Lehre im Sommersemester 2024”}
    };
    %------------ Header ---------------------
    \node[fancytitle, right=10pt] at (box.north west) {Abgeschlossene Semester taggen};
 \end{tikzpicture}
\smallskip

\begin{tikzpicture}
    \node [mybox, align=left] (box){%
     alle Tags pushen:\\\smallskip\texttt{git push origin –tags} \\nur bestimmten Tag pushen:\\\texttt{git push origin SoSe24}
    };
    %------------ Header ---------------------
    \node[fancytitle, right=10pt] at (box.north west) {Tags auf den Server übertragen};
 \end{tikzpicture}
\smallskip

\hypertarget{allgemein-9}{%
\section{Allgemein}\label{allgemein-9}}

\begin{tikzpicture}
    \node [mybox, align=left] (box){%
     Ins Verzeichnis wechseln\\\texttt{git init}
    };
    %------------ Header ---------------------
    \node[fancytitle, right=10pt] at (box.north west) {Neues Git Repository};
 \end{tikzpicture}
\smallskip

\hypertarget{krefeld-9}{%
\subsection{Krefeld}\label{krefeld-9}}

\begin{tikzpicture}
    \node [mybox, align=left] (box){%
     Neuen Branch anlegen:\\\texttt{git checkout -b SoSe2024}\\ \texttt{git push –set-upstream origin SoSe2024}
    };
    %------------ Header ---------------------
    \node[fancytitle, right=10pt] at (box.north west) {Ein neues Semester};
 \end{tikzpicture}
\smallskip

\begin{tikzpicture}
    \node [mybox, align=left] (box){%
     \texttt{git checkout main}\\ \texttt{git merge SoSe2024}
    };
    %------------ Header ---------------------
    \node[fancytitle, right=10pt] at (box.north west) {Semester in \texttt{main} mergen};
 \end{tikzpicture}
\smallskip

\begin{tikzpicture}
    \node [mybox, align=left] (box){%
     Tag setzen:\\\texttt{git tag -a SoSe24 -m “Meine Lehre im Sommersemester 2024”}
    };
    %------------ Header ---------------------
    \node[fancytitle, right=10pt] at (box.north west) {Abgeschlossene Semester taggen};
 \end{tikzpicture}
\smallskip

\begin{tikzpicture}
    \node [mybox, align=left] (box){%
     alle Tags pushen:\\\smallskip\texttt{git push origin –tags} \\nur bestimmten Tag pushen:\\\texttt{git push origin SoSe24}
    };
    %------------ Header ---------------------
    \node[fancytitle, right=10pt] at (box.north west) {Tags auf den Server übertragen};
 \end{tikzpicture}
\smallskip

\hypertarget{allgemein-10}{%
\section{Allgemein}\label{allgemein-10}}

\begin{tikzpicture}
    \node [mybox, align=left] (box){%
     Ins Verzeichnis wechseln\\\texttt{git init}
    };
    %------------ Header ---------------------
    \node[fancytitle, right=10pt] at (box.north west) {Neues Git Repository};
 \end{tikzpicture}
\smallskip

\hypertarget{krefeld-10}{%
\subsection{Krefeld}\label{krefeld-10}}

\begin{tikzpicture}
    \node [mybox, align=left] (box){%
     Neuen Branch anlegen:\\\texttt{git checkout -b SoSe2024}\\ \texttt{git push –set-upstream origin SoSe2024}
    };
    %------------ Header ---------------------
    \node[fancytitle, right=10pt] at (box.north west) {Ein neues Semester};
 \end{tikzpicture}
\smallskip

\begin{tikzpicture}
    \node [mybox, align=left] (box){%
     \texttt{git checkout main}\\ \texttt{git merge SoSe2024}
    };
    %------------ Header ---------------------
    \node[fancytitle, right=10pt] at (box.north west) {Semester nach Abschluss in \texttt{main} mergen};
 \end{tikzpicture}
\smallskip

\begin{tikzpicture}
    \node [mybox, align=left] (box){%
     Tag setzen:\\\texttt{git tag -a SoSe24 -m “Meine Lehre im Sommersemester 2024”}
    };
    %------------ Header ---------------------
    \node[fancytitle, right=10pt] at (box.north west) {Abgeschlossene Semester taggen};
 \end{tikzpicture}
\smallskip

\begin{tikzpicture}
    \node [mybox, align=left] (box){%
     alle Tags pushen:\\\smallskip\texttt{git push origin –tags} \\nur bestimmten Tag pushen:\\\texttt{git push origin SoSe24}
    };
    %------------ Header ---------------------
    \node[fancytitle, right=10pt] at (box.north west) {Tags auf den Server übertragen};
 \end{tikzpicture}
\smallskip

\hypertarget{allgemein-11}{%
\section{Allgemein}\label{allgemein-11}}

\begin{tikzpicture}
    \node [mybox, align=left] (box){%
     Ins Verzeichnis wechseln\\\texttt{git init}
    };
    %------------ Header ---------------------
    \node[fancytitle, right=10pt] at (box.north west) {Neues Git Repository};
 \end{tikzpicture}
\smallskip

\hypertarget{krefeld-11}{%
\subsection{Krefeld}\label{krefeld-11}}

\begin{tikzpicture}
    \node [mybox, align=left] (box){%
     Neuen Branch anlegen:\\\texttt{git checkout -b SoSe2024}\\ \texttt{git push –set-upstream origin SoSe2024}
    };
    %------------ Header ---------------------
    \node[fancytitle, right=10pt] at (box.north west) {Ein neues Semester};
 \end{tikzpicture}
\smallskip

\begin{tikzpicture}
    \node [mybox, align=left] (box){%
     \texttt{git checkout main}\\ \texttt{git merge SoSe2024}
    };
    %------------ Header ---------------------
    \node[fancytitle, right=10pt] at (box.north west) {Semester nach Abschluss in \texttt{main} mergen};
 \end{tikzpicture}
\smallskip

\begin{tikzpicture}
    \node [mybox, align=left] (box){%
     Tag setzen:\\\texttt{git tag -a SoSe24 -m “Meine Lehre im Sommersemester 2024”}
    };
    %------------ Header ---------------------
    \node[fancytitle, right=10pt] at (box.north west) {Abgeschlossene Semester taggen};
 \end{tikzpicture}
\smallskip

\begin{tikzpicture}
    \node [mybox, align=left] (box){%
     alle Tags pushen:\\\smallskip\texttt{git push origin –tags} \\nur bestimmten Tag pushen:\\\texttt{git push origin SoSe24}
    };
    %------------ Header ---------------------
    \node[fancytitle, right=10pt] at (box.north west) {Tags auf den Server übertragen};
 \end{tikzpicture}
\smallskip

\hypertarget{allgemein-12}{%
\section{Allgemein}\label{allgemein-12}}

\begin{tikzpicture}
    \node [mybox, align=left] (box){%
     Ins Verzeichnis wechseln\\\texttt{git init}
    };
    %------------ Header ---------------------
    \node[fancytitle, right=10pt] at (box.north west) {Neues Git Repository};
 \end{tikzpicture}
\smallskip

\hypertarget{krefeld-12}{%
\subsection{Krefeld}\label{krefeld-12}}

\begin{tikzpicture}
    \node [mybox, align=left] (box){%
     Neuen Branch anlegen:\\\texttt{git checkout -b SoSe2024}\\ \texttt{git push –set-upstream origin SoSe2024}
    };
    %------------ Header ---------------------
    \node[fancytitle, right=10pt] at (box.north west) {Ein neues Semester};
 \end{tikzpicture}
\smallskip

\begin{tikzpicture}
    \node [mybox, align=left] (box){%
     \texttt{git checkout main}\\ \texttt{git merge SoSe2024}
    };
    %------------ Header ---------------------
    \node[fancytitle, right=10pt] at (box.north west) {Semester nach Abschluss in \texttt{main} mergen};
 \end{tikzpicture}
\smallskip

\begin{tikzpicture}
    \node [mybox, align=left] (box){%
     Tag setzen:\\\texttt{git tag -a SoSe24 -m “Meine Lehre im Sommersemester 2024”}
    };
    %------------ Header ---------------------
    \node[fancytitle, right=10pt] at (box.north west) {Abgeschlossene Semester taggen};
 \end{tikzpicture}
\smallskip

\begin{tikzpicture}
    \node [mybox, align=left] (box){%
     alle Tags pushen:\\\smallskip\texttt{git push origin –tags} \\nur bestimmten Tag pushen:\\\texttt{git push origin SoSe24}
    };
    %------------ Header ---------------------
    \node[fancytitle, right=10pt] at (box.north west) {Tags auf den Server übertragen};
 \end{tikzpicture}
\smallskip



%--- ENDE ------------
\end{multicols*}\end{document}
